
\iffalse
\subsection*{Cluster of outliers}\label{sec3.1}  

We collected outlier behaviours as indicated by the two methods above and ran a KMeans clustering. We found 881 outliers and 8 clusters using outliers clustering method and 573 outliers and 7 clusters using SI method. 

\begin{table}[!htbp]
\begin{adjustbox}{max width=1.2\textwidth}
\begin{tabular}{lllll}
\hline
Size &
  SI outliers &
  \% &
  KMeans outliers &
  \% \\ \hline
1 &
  \begin{tabular}[c]{@{}l@{}}Increased Steps Morning; \\ reduced Steps Afternoon and Evening.\end{tabular} &
  31.06 &
  Increased HRV and Steps x1000. &
  (17.48\%): \\
2 &
  \begin{tabular}[c]{@{}l@{}}Reduced Steps Morning and Afternoon; \\ increased Steps Evening.\end{tabular} &
  17.45 &
  \begin{tabular}[c]{@{}l@{}}Significantly reduced TST; \\ significantly increased Midsleep.\end{tabular} &
  (17.14\%): \\
3 &
  \begin{tabular}[c]{@{}l@{}}Significantly increased HRV; \\ slightly reduced Steps Evening.\end{tabular} &
  12.74 &
  \begin{tabular}[c]{@{}l@{}}Slightly reduced HRV; \\ slight decrease in TST; slight increase in Midsleep.\end{tabular} &
  (15.21\%) \\
4 &
  \begin{tabular}[c]{@{}l@{}}Reduced HRV and TST; \\ increased Midsleep.\end{tabular} &
  12.57 &
  \begin{tabular}[c]{@{}l@{}}Significant increase in morning steps; \\ decreased afternoon and evening steps.\end{tabular} &
  14.19\% \\
5 &
  \begin{tabular}[c]{@{}l@{}}Reduced Steps Morning, Afternoon, and Evening; \\ increased Steps Night.\end{tabular} &
  11.34 &
  \begin{tabular}[c]{@{}l@{}}Markedly reduced morning and afternoon steps; \\ significantly increased evening steps.\end{tabular} &
  11.58\% \\
6 &
  Increased Steps x1000; reduced TST and Midsleep. &
  8.55 &
  \begin{tabular}[c]{@{}l@{}}Reduced HRV; significantly increased TST; \\ decreased Midsleep.\end{tabular} &
  8.85\% \\
7 &
  \begin{tabular}[c]{@{}l@{}}Increased Steps Afternoon, TST, and Midsleep; \\ reduced Steps Morning and Evening.\end{tabular} &
  6.28 &
  \begin{tabular}[c]{@{}l@{}}Reduced afternoon and evening steps; notably reduced TST; \\ significantly decreased Midsleep.\end{tabular} &
  8.51\% \\
8 &
  - &
  - &
  Extremely increased night steps. &
  7.04\% \\ \hline
\end{tabular}
\end{adjustbox}
\caption{corona: Cluster characteristics of SI and KMeans method}
\end{table}

The following cluster characteristics appear in both methods: (1) Reduced HRV, increased TST, decreased Midsleep, (2) Increased Steps Morning; reduced Steps Afternoon and Evening, (3) Reduced Steps Morning and Afternoon; increased Steps Evening, and (4) Reduced Steps Morning, Afternoon, and Evening; increased Steps Night. Outliers can be categorized to good behaviour (Increased step count, increased HRV) and bad behaviour (reduced HRV, reduced TST and later sleep).   


We repeated the same procedure for the MoMo-Mood study and identified (457 outliers and 5 clusters) using baseline clustering method and (153 outliers and 8 clusters) using SI method.

% Please add the following required packages to your document preamble:
% \usepackage{booktabs}
\begin{table}[!htbp]
\begin{adjustbox}{max width=1.2\textwidth}
\begin{tabular}{@{}lllll@{}}
\toprule
Size &
  SI outliers &
  \% &
  KMeans outliers &
  \% \\ \midrule
1 &
  \begin{tabular}[c]{@{}l@{}}Reduced time at home, increased acceleration in \\ some parts of the day\end{tabular} &
  60 &
  Increased acceleration in the afternoon and evening &
  26.04 \\
2 &
  Increased acceleration in the evening &
  13.33 &
  \begin{tabular}[c]{@{}l@{}}Reduced time at home, increased entropy; reduced \\ acceleration in the morning and afternoon\end{tabular} &
  23.41 \\
3 &
  \begin{tabular}[c]{@{}l@{}}Reduced time at home, increased acceleration in\\ the morning and afternoon\end{tabular} &
  12.57 &
  Increased acceleration in the evening &
  22.98 \\
4 &
  \begin{tabular}[c]{@{}l@{}}Reduced location entropy, increased accel std\\ in the afternoon and evening\end{tabular} &
  11.85 &
  \begin{tabular}[c]{@{}l@{}}Reduced time at home, increased acceleration in\\ the morning and afternoon\end{tabular} &
  17.72 \\
5 &
   &
   &
  \begin{tabular}[c]{@{}l@{}}Reduced location entropy, increased accel std\\ in the afternoon and evening\end{tabular} &
  9.85 \\ \hline
\end{tabular}
\end{adjustbox}
\caption{Momo: Cluster characteristics of SI and KMeans method}
\end{table}

The outliers mainly concern days when the subjects exhibited different mobility patterns than normal.
\fi 

\subsection*{Atypical behaviours}

- Atypical behaviours with regards to 2 routines (screen, sleep) appear to be consistent across studies.

- MoMo: No clear seperation in atypical behaviours between control and patients

\subsection*{Adherence to various routine types}

- MoMo: Control tend to maintain more regular routines than patients (p=.02).

- 

\textbf{Remarks: corona}

- Outliers do not always happen in consecutive days. Relaxing the conditions to identify outliers increases the amount of outlier data and reduces the average distance between consecutive days with outliers. Std 2 -> 33  days. Std 1.85 -> 22 days. Std 1.5 -> 17 days.

- The maximum outliers clusters detected are 8 (found at std 1.5).

- Overlap between 2 baseline methods: 55\% of SI outliers lie in cluster outliers. This hold true regardless of the threshold chosen.


\textbf{Differences between outlier cluster profiles}

\begin{figure}[!htbp]
    \centering
    \includegraphics[width=1\linewidth]{Screenshot 2024-07-04 at 13.51.44.png}
    \caption{Cluster characteristics}
    \label{fig:enter-label}
\end{figure}

- Age profile difference: older people tend to do more exercise than normal, resulting in higher HRV and steps count. Younger people on the other hand tend to experience bad behaviour (low HRV, low TST, later sleep).

- Stress score difference: good outlier behaviours are associated with improved stress and sleep quality score compared to bad behaviour.